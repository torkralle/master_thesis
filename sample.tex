\documentclass[a4paper,11pt]{article}
\usepackage{luatexja}
\usepackage{graphicx}
\usepackage{amsmath}
\usepackage{siunitx}
\usepackage{float}

\title{単振り子の周期に関する実験}
\author{工学部物理学科 \\ 学籍番号:12345}
\date{\today}

\begin{document}
\maketitle

\section{実験目的}
単振り子の振動周期と糸の長さの関係を調べ、重力加速度を算出する。

\section{実験装置}
\begin{itemize}
    \item 糸(長さ可変)
    \item 重り(質量:\SI{100}{\gram})
    \item ストップウォッチ
    \item 定規
\end{itemize}

\section{実験方法}
\begin{enumerate}
    \item 糸の長さを\SI{30}{\centi\metre}に設定
    \item 振幅約\SI{5}{\degree}で振動を開始
    \item 10周期の時間を測定
    \item 上記を3回繰り返し、平均値を算出
\end{enumerate}

\section{実験結果}
周期$T$と振り子の長さ$L$の関係は以下の式で表される:
\[T = 2\pi\sqrt{\frac{L}{g}}\]

\begin{table}[H]
\centering
\begin{tabular}{|c|c|c|}
\hline
長さ [\si{\centi\metre}] & 10周期 [\si{\second}] & 周期 [\si{\second}] \\
\hline
30 & 10.92 & 1.092 \\
\hline
\end{tabular}
\end{table}

\section{考察}
測定値から計算された重力加速度は\SI{9.81}{\metre\per\second\squared}となり、
理論値とよく一致している。

\end{document}
