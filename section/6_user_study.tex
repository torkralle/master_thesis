本研究では候補者の選定とインタビューを実施し、ユーザースタディから得られた示唆を整理した。候補者はカスタマーサービスと連携して13名を選定し、アンケートとインタビュー項目を用意した。現段階では30分程度のインタビューを2件実施した。

\subsection*{インタビュー対象者}
1件目のインタビュー対象者は、介護経験7年目のワーカーで、現在は副業としてマッチングサービスを月4〜5回程度利用している。過去に4つの事業所での勤務経験があり、サービスの活用方法について豊富な知見を持つ。

2件目のインタビュー対象者は、実務者研修時に日本福祉アカデミーでモンスケを知り、母親の利用をきっかけに単発バイトとして利用を開始した。複数のサービス機能を実際に利用しており、システムの使い勝手について具体的なフィードバックを提供した。

\subsection*{求人選択の実態}
インタビューから、ワーカーの求人選択プロセスには明確な優先順位があることが判明した。

\paragraph{リピート志向の強さ}
ワーカーは一度良いと感じた施設には繰り返し応募する傾向が非常に強い。「いつも行くところがあれば、そこを選ぶ」という発言から、リピート施設が第一選択肢となっており、新規施設の探索は「いつもの施設が空いていない場合」に限られることが分かった。この背景には、「お互いゼロから教えなくていい」という相互の効率性への配慮がある。

\paragraph{駐車場の重要性}
予想外に高い重要度を示したのが「駐車場の有無」である。交通費が往復500円で、地下鉄の往復料金を超えると赤字になるため、駐車場がない施設は敬遠される。これは時給以上に意思決定に影響する要因であることが示された。

\paragraph{業務内容の選好}
特定の業務(入浴介助等)の有無は、時給よりも優先される場合がある。「入浴がない」ことが施設選択の重要な条件となっており、業務内容の詳細表示が求人の魅力度に直結することが確認された。

\paragraph{給与の基準}
給与については「8時間勤務、1時間休憩、交通費込みで1万円程度」が望ましい水準として挙げられた。時給換算よりも「1回でがっつり稼ぎたい」という総額志向が見られた。

\subsection*{情報収集の実態}
新規施設を選ぶ際、ワーカーは以下の手順で情報を収集していた:
\begin{enumerate}
\item 近い場所から順に閲覧
\item 業務内容と口コミを確認
\item 知人からの横のつながりで得た情報を参照
\end{enumerate}

特に「モンスケをやっている友達」からの情報共有が活発であり、「ここはヤバかった」といった具体的な評判が意思決定に強く影響することが判明した。

\subsection*{サービスへの評価}
現行システムに対しては高い満足度が示された。特にLINEでのカスタマーサポートについて「電話よりも自分のタイミングで返せる」点が評価されている。機能面での大きな不満はなく、細かいバグ以外は現状に満足しているとのことであった。

2件目のインタビューからは、具体的な機能への評価が得られた。レビュー機能により「役に立てた実感」が得られること、QRコード読み取り時の励ましメッセージ、即日出金システムが学生など急な出費に対応できること、詳細な地図情報や採用情報の掲載が丁寧であることなどが評価された。

一方で、改善点として以下が指摘された:
\begin{itemize}
\item 月末の土日勤務時の即日出金処理の翌月持ち越しに関する不明点
\item アプリ内ウォレット残高からの即日出金依頼への対応遅延
\item アプリとLINEの両方で連絡する二度手間
\end{itemize}

\subsection*{オンボーディングの課題}
介護現場特有の課題として、初回訪問時のオンボーディングの難しさが指摘された。認知症患者や特別な配慮が必要な利用者の情報、施設内の場所などの説明が必要だが、一回限りの訪問者に詳細な説明をすることへの疑問が事業所側から生じている。この課題に対して、以下の改善案が提示された:
\begin{itemize}
\item 事業所インタビュー記事のアプリ内閲覧機能
\item 動画解説による初回応募ハードルの低減
\item キャンセル発生時の「駆けつけ隊」機能
\item ゲーミフィケーション要素の追加(時間経過で報酬が変動するなど)
\end{itemize}

これらの提案は、単発マッチングにおける情報非対称性の解消と、ワーカーの迅速な現場適応を支援するための具体的な方向性を示している。
