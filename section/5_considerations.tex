本研究の実験結果から導かれる主要な考察を以下に示す。

\begin{itemize}
\item 多く働いているユーザー(データが豊富なユーザー)とそうでないユーザーで応募傾向が大きく異なる。頻繁に利用するユーザーは特定の求人に繰り返し応募する傾向があり、これをそのまま学習すると偏りが生じる可能性がある。
\item 学習データ量については、15件〜20件程度が最適帯に見え、それを超えると過去の嗜好が現在の嗜好を汚染し精度が低下する傾向があった。したがって、時間的な重み付け(古いデータを低減する等)を導入することが有効である。
\item TF-IDFは計算コストが低く、タイトルや短文の比較ではBERTに匹敵する実用的な精度を示す場合がある。モデル選択は精度と計算コストのトレードオフを考慮して行うべきである。
\end{itemize}

提案:ユーザーの利用頻度に基づく重み付け、時間的重み付け、およびTF-IDFとBERTを組み合わせたハイブリッド戦略を検討することが望ましい。
