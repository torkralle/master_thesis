本研究の実験結果から導かれる主要な考察を以下に示す。

\begin{itemize}
\item \textbf{利用頻度による精度差:}多く働いているユーザー(データが豊富なユーザー)とそうでないユーザーで応募傾向が大きく異なる。頻繁に利用するユーザー(ワーカーA、B)は100\%の精度で予測できるのに対し、利用頻度が低いユーザー(ワーカーI、J)では複合スコアが0.14〜0.17と大幅に低下する。これはユーザーの利用頻度に応じた重み付け調整の必要性を示唆する。

\item \textbf{最適学習データ量:}学習データ量については、15件〜20件程度が最適帯であることが判明した。20件学習時の最高複合スコアは0.7112(均等パターン)であるのに対し、30件学習時は0.5430(総合重視パターン)まで低下する。これは約24\%の精度低下に相当する。過度に過去のデータを学習すると、古い嗜好が現在の嗜好を汚染し精度が低下するためと考えられる。したがって、時間的な重み付け(古いデータを低減する等)を導入することが有効である。

\item \textbf{重み付けパターンの選択:}20件学習では「均等(equal)」パターンが最良であったが、30件学習では「総合重視(comprehensive)」が最良となる。これは学習データ量によって最適な重み付けが変化することを示しており、動的な重み調整が望ましい。

\item \textbf{TF-IDFの実用性:}実験3においてTF-IDFは計算コストが低く、タイトルや短文の比較ではBERTに匹敵する実用的な精度(1.0)を示した。リアルタイム性が要求されるUXでは、TF-IDFによる高速フィルタリング後にBERTで精密ランキングを行うハイブリッド戦略が有効である。
\end{itemize}

提案:ユーザーの利用頻度に基づく重み付け、時間的重み付け(直近15〜20件を重視)、およびTF-IDFとBERTを組み合わせたハイブリッド戦略を検討することが望ましい。
