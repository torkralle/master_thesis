本研究の実験結果から導かれる主要な考察を以下に示す。

\begin{itemize}
\item \textbf{利用頻度による精度差:}多く働いているユーザー(データが豊富なユーザー)とそうでないユーザーで応募傾向が大きく異なる。頻繁に利用するユーザー(ワーカーA、B)は100\%の精度で予測できるのに対し、利用頻度が低いユーザー(ワーカーI、J)では複合スコアが0.14〜0.17と大幅に低下する。これはユーザーの利用頻度に応じた重み付け調整の必要性を示唆する。

\item \textbf{最適学習データ量:}学習データ量については、15件〜20件程度が最適帯であることが判明した。20件学習時の最高複合スコアは0.7112(均等パターン)であるのに対し、30件学習時は0.5430(総合重視パターン)まで低下する。これは約24\%の精度低下に相当する。過度に過去のデータを学習すると、古い嗜好が現在の嗜好を汚染し精度が低下するためと考えられる。したがって、時間的な重み付け(古いデータを低減する等)を導入することが有効である。

\item \textbf{重み付けパターンの選択:}20件学習では「均等(equal)」パターンが最良であったが、30件学習では「総合重視(comprehensive)」が最良となる。これは学習データ量によって最適な重み付けが変化することを示しており、動的な重み調整が望ましい。

\item \textbf{TF-IDFの実用性:}実験3においてTF-IDFは計算コストが低く、タイトルや短文の比較ではBERTに匹敵する実用的な精度(1.0)を示した。リアルタイム性が要求されるUXでは、TF-IDFによる高速フィルタリング後にBERTで精密ランキングを行うハイブリッド戦略が有効である。

\item \textbf{リピート行動の重要性:}ユーザースタディから、ワーカーは一度良いと感じた施設に繰り返し応募する傾向が非常に強いことが判明した。「いつもの施設」が第一選択肢となり、新規施設の探索は限定的である。これは本実験1で観察された「多く働いているユーザーは特定の求人に繰り返し応募する」という傾向と一致しており、過去の応募履歴による推薦の有効性を裏付ける。

\item \textbf{非テキスト要因の影響:}ユーザースタディでは、駐車場の有無や特定業務(入浴介助等)の可否が、時給以上に意思決定に影響することが示された。これらの構造化データは、本研究で提案した7項目の重み付けにおいて「勤務条件」として扱われているが、さらに細分化して高い重みを与える必要がある。特に「駐車場」は往復交通費が赤字になる閾値(500円)との関係で重視されており、経済的合理性に基づく意思決定が行われている。

\item \textbf{総額志向と表示方法:}給与については時給換算よりも「1回でがっつり稼ぎたい」という総額志向が見られた。これはUIにおいて時給だけでなく、想定総収入を明示することの重要性を示唆する。

\item \textbf{口コミと横のつながり:}知人からの情報共有が活発であり、「ここはヤバかった」といった具体的な評判が意思決定に強く影響する。これは本研究のスコアリングモデルには含まれていない要素であり、今後は評価スコアやレビューデータを統合することが課題である。

\item \textbf{フィードバック機能の心理的効果:}レビュー機能やQRコード読み取り時のメッセージは、単なる確認手段以上の価値を持つ。「役に立てた実感」というワーカーの内発的動機付けを高め、サービス継続利用につながる可能性が示された。

\item \textbf{オンボーディングの情報非対称性:}単発マッチングにおける最大の課題は、初回訪問時の情報格差である。動画解説や事業所紹介記事など、事前情報提供の充実が応募ハードルを下げ、マッチング成功率を向上させる可能性がある。

\item \textbf{マルチチャネル対応の二重性:}アプリとLINEの両方で連絡する二度手間が指摘された一方で、LINEサポートは高く評価されている。この矛盾は、チャネル統合ではなく、各チャネルの役割を明確化する設計が重要であることを示唆している。
\end{itemize}

提案:ユーザーの利用頻度に基づく重み付け、時間的重み付け(直近15〜20件を重視)、およびTF-IDFとBERTを組み合わせたハイブリッド戦略を検討することが望ましい。さらに、ユーザースタディで得られた知見を反映し、駐車場の有無や特定業務の可否に対して高い重みを設定し、総収入の明示や口コミスコアの統合を行うべきである。加えて、オンボーディングの充実(動画・記事)、フィードバック機能の継続的改善、マルチチャネルの役割明確化が、ユーザーエクスペリエンス向上の鍵となる。
