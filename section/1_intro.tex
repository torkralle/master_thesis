本研究は、介護特化型マッチングサービスにおけるマッチング精度の向上を目的とする。背景として、2040年に向けた介護人材不足が深刻化しており、厚生労働省の推計では約69万人の人手不足が見込まれている。一方で、有資格者でありながら介護職に従事していない「潜在介護士」が約37万人存在し、供給可能な人材が十分に活用されていない。

とくに本研究の対象としている北海道は高齢化の進行と人口偏在が顕著であり、2025年には高齢化率が約33.5\%に達すると予想される。都市部に人口が集中する一方で、広大な地域に点在する市町村では移動距離や通勤手段の制約が大きく、マッチングの難易度が高い。

これらの課題を踏まえ、本研究では単純な定型条件マッチ(勤務地・日程・資格等)だけでなく、テキストの意味的類似性や履歴情報、時間的要因を組み合わせた重み付けにより、ワーカーと求人の潜在的親和性をより高精度に評価する手法を提案する。
