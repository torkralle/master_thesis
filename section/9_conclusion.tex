\section*{結論}
本研究では、介護マッチングにおける意味的類似性の導入と複数項目を組み合わせた重み付けが、実務的に有用なマッチング精度を達成し得ることを示した。予備実験では日本語Sentence-BERTが高い分離度を示したが、実装上のトレードオフを鑑みるとTF-IDFも有力である。

実験では学習データ量や重み付けパターンにより精度が変動し、特に15件〜20件程度の学習データで良好な結果が得られた。これは過去データの扱い(時間的重み付け)の重要性を示唆する。

\section*{付録:行列間コサイン類似度}
行列(あるいは文書)をベクトル化し、コサイン類似度を用いて類似度を評価する。コサイン類似度は以下で定義される:

\[
\cos(\theta)=\frac{\mathbf{a}\cdot\mathbf{b}}{\|\mathbf{a}\|\,\|\mathbf{b}\|}=\frac{\sum_i a_i b_i}{\sqrt{\sum_i a_i^2}\,\sqrt{\sum_i b_i^2}}
\]

ここで\(\mathbf{a}\)、\(\mathbf{b}\)はベクトル化された行列(または文書)を表す。
