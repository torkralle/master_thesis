\section*{結論}
本研究では、介護マッチングにおける意味的類似性の導入と複数項目を組み合わせた重み付けが、実務的に有用なマッチング精度を達成し得ることを示した。予備実験では日本語Sentence-BERTが高い分離度を示したが、実装上のトレードオフを鑑みるとTF-IDFも有力である。

実験では学習データ量や重み付けパターンにより精度が変動し、特に15件〜20件程度の学習データで良好な結果が得られた。これは過去データの扱い(時間的重み付け)の重要性を示唆する。

本手法により、ワーカーの希望とより適合した求人を効率的に提示できるシステムの基盤が構築された。今後は、実データを用いたさらなる検証と、複数の手法を組み合わせたより高精度なマッチングアルゴリズムの開発、実サービスへの実装とユーザーフィードバックに基づく改善サイクルの確立が課題である。
