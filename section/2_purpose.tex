本研究の目的は、ワーカーと求人のマッチングにおける「意味的な親和性」を捉え、ワーカー側・事業者側双方の検索・推薦の手間を最小化することである。従来のマッチングは主に勤務地・日程・資格等の定型条件で行われており、希望職種や過去の職務経験と求人内容の意味的な一致は十分に評価されていない。

その結果、潜在的に適合するワーカーと求人が見逃されることがある。また、ワーカーが自ら細かい条件を指定して検索する手間が大きい点も課題である。本研究では自然言語処理によるベクトル化、履歴情報、時間的重み付けなどを組み合わせることで、これらの課題を解決する手法を検討する。
