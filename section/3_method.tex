本研究では以下の要素を組み合わせた重み付けを提案する。

\section*{自然言語処理モデルを利用した距離計算}
求人タイトルや業務内容、職務経歴などのテキストを自然言語処理モデルでベクトル化し、文書間の距離(類似度)を計算する。これにより職種的距離やスキル的距離を定量化できる。

\subsection*{ベクトル化とコサイン類似度}
自然言語処理モデル(BERTやTF-IDF等)を用いて、テキストを高次元ベクトル空間に埋め込む。具体的には、求人タイトルや業務内容などの各テキストフィールドをモデルに入力し、固定長のベクトル表現を得る。

テキスト間の類似度評価には、ベクトル化された表現間のコサイン類似度を用いる。2つのベクトル\(\mathbf{a}\)と\(\mathbf{b}\)のコサイン類似度は以下で定義される:

\[
\cos(\theta)=\frac{\mathbf{a}\cdot\mathbf{b}}{\|\mathbf{a}\|\,\|\mathbf{b}\|}=\frac{\sum_{i=1}^{n} a_i b_i}{\sqrt{\sum_{i=1}^{n} a_i^2}\,\sqrt{\sum_{i=1}^{n} b_i^2}}
\]

ここで、\(\mathbf{a}=(a_1, a_2, \ldots, a_n)\)および\(\mathbf{b}=(b_1, b_2, \ldots, b_n)\)は\(n\)次元ベクトル、\(\mathbf{a}\cdot\mathbf{b}\)は内積、\(\|\mathbf{a}\|\)および\(\|\mathbf{b}\|\)はそれぞれのベクトルのユークリッドノルムを表す。

コサイン類似度は\([-1, 1]\)の範囲の値を取り、1に近いほど類似度が高く、0は直交(無関係)、-1は完全に反対の方向を示す。テキスト解析では通常、非負の値域\([0, 1]\)で扱われる。

\subsection*{行列間の類似度計算}
複数のテキストフィールド(例:求人タイトル、業務内容、介護項目など)を持つ文書全体の類似度を計算する場合、各フィールドをベクトル化した後、それらを行列として扱う。

文書Aのフィールドベクトルを\(\mathbf{a}_1, \mathbf{a}_2, \ldots, \mathbf{a}_m\)、文書Bのフィールドベクトルを\(\mathbf{b}_1, \mathbf{b}_2, \ldots, \mathbf{b}_m\)とする。このとき、行列間の類似度は以下の手順で計算される:

\begin{enumerate}
\item 各フィールドペア\((\mathbf{a}_i, \mathbf{b}_i)\)のコサイン類似度\(s_i\)を計算
\item 各フィールドに重み\(w_i\)を付与し、加重平均を求める:
\[
\text{Similarity}(A, B) = \frac{\sum_{i=1}^{m} w_i \cdot s_i}{\sum_{i=1}^{m} w_i}
\]
\end{enumerate}

この手法により、複数の観点から文書間の類似性を総合的に評価できる。重み\(w_i\)は実験的に調整され、本研究では7つの項目(求人タイトル、業務内容、介護項目、施設、勤務時間、時給、勤務日)に対して様々な重み付けパターンを検証した。

\section*{時間的距離と勤務条件の差分}
勤務地や勤務時間、勤務日などの構造化された条件に基づき時間的・地理的な差分を計算し、ベクトル類似度と組み合わせることで総合スコアを算出する。

\section*{過去応募履歴の活用}
過去の応募履歴や承認履歴を用いてユーザーの志向性をモデル化する。学習データ(過去N件)から平均ベクトルを算出し、新しい求人との類似度を評価する手法を本研究では実験的に採用した。
